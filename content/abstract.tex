\selectlanguage{german}

\subsubsection*{Zusammenfassung}
\label{sec:Zusammenfassung}

Diese Bachelor-Arbeit zielt darauf ab einen
Stickstoffmonoxid-zu-Stickstoffdioxid-Konverter weiter zu
verbessern. Dieser soll zusammen mit einem \ch{NO2} ICAD
Messinstrument verwendet werden, um die Stickoxid-Belastung in
städtischen Gebieten zu bestimmen. Der Hauptbestandteil dieses
Konverters ist ein Ozon-Generator, der mit Umgebungsluft versorgt
wird. In dieser Arbeit ist es gelungen die ozonhaltige Luft auf eine
vernachlässigbare Konzentration von Stickstoffdioxid (\ch{NO2}) zu
säubern, was die Verwendung in einem Konverter erst ermöglichte. Im
Anschluß, wurde dieser charakteriesiert und die mögliche \ch{NO}
Messgenauigkeit im Laborsystem bestimmt. Zusätzlich wurden erste
Anwendungen für Fahrzeugemissionsmessungen im Stadtgebiet Heidelberg
getestet. Es sind immer noch weiter Untersuchungen nötig um das
Potential des Konverters voll auszuschöpfen. Insbesondere muss das das
Adsorptionsverhalten von Ozon an den Teflonschläuchen besser
verstanden werden um weitere pragmatischere Messmethoden zu
ermöglichen.

\selectlanguage{english}

\subsubsection*{Abstract}
\label{sec:abstract}

This bachelor thesis aims at a further improvement of a nitrogen
monoxide (\ch{NO}) to nitrogen dioxide (\ch{NO2}) converter, which
will be used in conjunction with an \ch{NO2} ICAD instrument to
measure the nitrogen oxide (\ch{NO_x}) pollution in urban areas. The
main component of this converter is an ozone generator. This work
succeeded in making its output \ch{NO2} free, thus making its use in
the converter possible. Next, a characterisation of the converter was
performed and the possible \ch{NO} measurement accuracy was
determined. Additionally, first productive applications were tested by
performing live vehicle measurements in Heidelberg. There are still
further investigations necessary to reach the full potential of this
setup. First and foremost, the adsorption behaviour of ozone at the
teflon walls of the tubes has to be investigated in order to allow for
more practical measurement modes.

%%% Local Variables: 
%%% mode: latex
%%% TeX-master: "../Bachelor"
%%% End: 
