
\subsubsection*{Zusammenfassung}
\label{sec:Zusammenfassung}

Das Ziel dieser Arbeit ist es das nötige Handwerkszeug zu
entwickeln, um den Uniformisierungssatz für kompakte Riemannsche
Flächen beweisen zu können. Der Satz besagt, dass sich diese
Flächen in drei Typen unterteilen lassen: zum einen die Riemannsche
Zahlenkugel $\P^1$, die Flächen, die sich aus $\C$ modulo der
Wirkung eines Gitters $\Gamma$ ergeben und die Flächen, die sich aus
$\h$ modulo der Wirkung einer Fuchsschen Gruppe
ergeben. Diese Charakterisierung ist immer bis auf Konforme
Äquivalenz zu verstehen.

Dieses Resultat ist tiefliegend, weswegen zunächst viele der
Konzepte und Ergebnisse aus der Funktionentheorie 1 auf
Riemannsche Flächen verallgemeinert werden müssen. Zu diesen
Ergebnissen gehört der (große) Riemannsche Abbildungssatz und der
Weierstraßsche Produktsatz.

Leider kann die Herleitung des Uniformisierungssatzes im Rahmen
einer Bachelor"=Arbeit nicht vollständig geführt werden, so dass
Grundlagen in der Überlagerungstheorie und Garbenkohomologie
vorhanden sein müssen, um dieser Arbeit gewinnbringend folgen zu können.

\subsubsection*{Abstract}
\label{sec:abstract}

This thesis has the goal to develop the tools necessary to proof the
Uniformization theorem for compact Riemann surfaces. This theorem
states that there are up to conformal equivalence three types of
surfaces. First there is the Riemann sphere $\P^1$, next there are the
surfaces which are conformally equivalent to $\C$ modulo the action
af a lattice $\Gamma$ and then there are the surfaces which are
conformally equivalent to $\h$ modulo the action of a Fuchsian group.


This result is far from trivial. Therefore the main part of this
thesis consists in the struggle to generalize well-known concepts of
Complex Analysis 1 to Riemann surfaces. Two examples are the
Weierstraß factorization theorem and the (big) Riemann mapping theorem
(sometimes called Uniformization theorem, as well).

Since this is only a Bachelor thesis, we cannot hope to cover all the
material needed to proof the Uniformization theorem. Thus the inclined
reader should have a rudimentary understanding of the theory of
covering spaces and sheaf cohomology, both in the context of Riemann surfaces.

%%% Local Variables: 
%%% mode: latex
%%% TeX-master: "../Bachelor"
%%% End: 
