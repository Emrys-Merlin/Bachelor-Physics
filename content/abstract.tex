\selectlanguage{german}

\subsubsection*{Zusammenfassung}
\label{sec:Zusammenfassung}

Diese Bachelor-Arbeit zielt darauf ab einen
Stickstoffmonoxid-zu-Stickstoffdioxid-Konverter weiter zu
verbessern. Dieser soll zusammen mit einem CE-DOAS Messinstrument
verwendet werden, um die Stickoxid-Belastung in städtischen Gebieten
zu bestimmen. Der Hauptbestandteil dieses Konverters ist ein
Ozon-Generator. Es war uns möglich seine Luft völlig von
Stickstoffdioxid (\ch{NO2}) zu säubern, was seine Verwendung in dem
Konverter erst ermöglichte. Im Anschluß, charakterisierten wir eben
diesen und bestimmten die mögliche \ch{NO} Messgenauigkeit in
Laborsystemen. Zusätzlich testeten wir erste Anwendungen durch
Fahrzeugmessungen im Stadtgebiet Heidelberg. Es sind immer noch weiter
Untersuchungen nötig um das Potential des Konverters voll
auszuschöpfen. Insbesondere ist ein besseres Verständnis des
Adsorptionsverhaltens von Ozon an den Teflonschläuchen nötig, um
weitere pragmatischere Messmethoden zu ermöglichen.

\selectlanguage{english}

\subsubsection*{Abstract}
\label{sec:abstract}

This Bachelor Thesis aims at an further improvement of a nitrogen
monoxide (\ch{NO}) to nitrogen dioxide (\ch{NO2}) converter, which
will be used in conjunction with a CE-DOAS instrument to measure the
nitrogen oxide pollution in urban areas. The main component of this
converter is an ozone generator. We succeeded in making its output
\ch{NO2} free and thus making its use in the converter possible. We
then turn towards the characterisation of the converter and determine
the possible \ch{NO} measurement accuracy. Additionally, we tested
first productive applications by performing live vehicle measurements
in Heidelberg. There are still further investigations necessary to
reach the full potential of this setup. First and foremost the
adsorption behaviour of ozone at the teflon walls of the tubes has to
be determined in order to allow for more practical measurement modes.

%%% Local Variables: 
%%% mode: latex
%%% TeX-master: "../Bachelor"
%%% End: 
