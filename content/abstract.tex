\selectlanguage{german}

\subsubsection*{Zusammenfassung}
\label{sec:Zusammenfassung}

Diese Bachelorarbeit hatte zur Aufgabe die Stickstoffmonoxidmessung
(\ch{NO}) mit einem bestehenden ICAD Messinstrument möglich
zumachen. Da \ch{NO} im Bereich der eingebaute blaue LED keinerlei
Absorption aufweist, Stickstoffdioxidmessungen (\ch{NO2}) aber sehr
gut möglich sind, wurde der Weg über eine indirekte Messung durch
einen \ch{NO}"-zu"-\ch{NO2}"-Konverter gewählt. Das Kernelement dieses
Konverters stellt ein Ozon-Generator dar, dessen Ausgangsluft komplett
von \ch{NO2} gereinigt werden musste. Diese Ziel konnte durch einen
nachgeschaltenen Silica Gel Filter vollständig realisiert werden. Im
Anschluss wurden die Eigenschaften des Aufbaus charakterisiert und
seine Messgenauigkeit bestimmt. In \ch{NO2}-freier synthetischer Luft
lag die Unsicherheit \ch{NO}-Konzentration im
Promillebereich. Messungen mit Umgebungsluft zeigten die Grenzen des
Instruments auf. Wird die selbe Cavity verwendet, muss abwechselnd der
Konverter an- und abgeschaltet werden, um eine Bestimmung der
\ch{NO}-Konzentration zu ermöglichen. Hierbei traten Prozesse zum
Vorschein, die eine lange Wartezeit nach dem Umschaltvorgang
erzwingen. Der Grund hierfür wird in der Ozonadsorption an der
Verschlauchung vermutet und bis diese nicht kompensiert werden können,
sollte vom produktiven Einsatz dieser alternierenden Messmethode
abgesehen werden. Dem ungeachtet konnten mit einer ersten
Fahrzeugmessung Anwendungen getestet werden. Dies war möglich, da für
diese Art von Messung kein Abschalten des Konverters nötig war. Durch
Vergleiche mit einer reinen \ch{NO2}-Cavity ergaben die Messungen,
dass die \ch{NO}-Werte einen hohen Einfluss auf die Luftbelastung im
Straßenverkehr haben und nicht einfach vernachlässigt werden können. 

\selectlanguage{english}

\subsubsection*{Abstract}
\label{sec:abstract}

This Bachelor Thesis aims at an further improvement of a nitrogen
monoxide (\ch{NO}) to nitrogen dioxide (\ch{NO2}) converter, which
will be used in conjunction with a ICAD instrument to measure the
nitrogen oxide (\ch{NO_x}) pollution in urban areas. The main
component of this converter is an ozone generator. I succeeded in
making its output \ch{NO2} free, thus making its use in the converter
possible. Next, a characterisation of the converter was performed and
the possible \ch{NO} measurement accuracy was
determined. Additionally, first productive applications are tested by 
performing live vehicle measurements in Heidelberg. There are still
further investigations necessary to reach the full potential of this
setup. First and foremost the adsorption behavior of ozone at the
teflon walls of the tubes has to be investigated in order to allow for
more practical measurement modes.

%%% Local Variables: 
%%% mode: latex
%%% TeX-master: "../Bachelor"
%%% End: 
