\section{Conclusion}
\label{sec:conclusion}

This Bachelor Thesis was dedicated to further improve a \ch{NO} to
\ch{NO2} converter by testing filtering capabilities of silica gel
behind the ozone generator. This was a complete success. The
additional \ch{NO2} signal at the generator output could be removed
completely, while still achieving over \SI{200}{ppm} Ozone
output. Additionally the startup time seems handable and lies in the
region of \SI{5}{\minute} after the start of the mercury lamp and the
pump of the ozone system. 

Next, I could show that in a synthetic setting without \ch{NO2} and
precisely controlled \ch{NO} the measurement deviation from the
computed \ch{NO} concentrations lies in the per mill region. Hence,
there is also a very high correlation between the measured and the
computed data. For measurements in ambient air it was necessary to
research the behaviour of our system after turning the converter on
and off alternatingly. During this investigation, I found a major
stumbling block. After turning off the ozone generator, there is still
residual ozone in the measurement system on a natural time scale of
\SI{2}{\minute} to \SI{2.5}{\minute}. This leads to an overestimation
of the determined \ch{NO2} values and thus complications in the
correct determination of the \ch{NO} values. The difficulty became
even more prominent during parallel measurements with a
chemiluminescence monitor, which showed that this alternating
measurement mode should not be used before the decay behaviour of the
ozone concentration is understood and can be compensated. In order to
do this, I think it is necessary to research the adsorption behaviour
of ozone at teflon walls. The residual ozone concentration seemed to
indicate a proportionality to the reaction tube length, which points
towards this explanation. If adsorption really is the reason, this
would have a deep impact on the design constraints of the
converter. So far a standard \SI{4}{\milli\meter} diameter teflon tube
was used as reaction path, which allowed for the necessary mixing of
the gases and which could easily be adapted to the necessary length of
\SI{10}{\meter}. In order to diminish the adsorption effects, we would
have to shorten the reaction path, while still keeping the dwell time
constant. This could be achieved by using a reaction tube with a
larger diameter, as this would reduce the surface area. However, the
question would have to be addressed of how to guarantee the mixing of
the ozone with the sample air. Since the used pumps only allow for
laminar flows in our system, it might be that the diameter becomes too
large for an adequate mixing by diffusion. A remedy could be the
introduction of one or multiple jets behind the ozone injection. This
would allow for turbulences and improved mixing, without unnessecarily
increasing the teflon surface. I think the adsorption reasearch and
possible alternative designs of the reaction path are the natural next
step in the development of the converter.

However, the adsorption effects have only to be taken into account, if
one wants to switch between \ch{NO_x} and \ch{NO2} measurements. If
one of the modes is fixed, there are no discernible adsorption
effects. As during vehicle measurements the actually important
quantity is the \ch{NO_x} concentration, the converter in the current
form is already ready for operation. Furthermore, comparing the cavity
data to the official data of the Heidelberg air quality measurement
station showed that using two cavities in parallel avoids the
difficulties introduced by the alternating measurement mode. The
vehicle results proved the necessity for a true \ch{NO_x} measurement
instrument, as the average emission of \ch{NO_x} lay a factor 10
higher than the emission of \ch{NO2}, but also here a few more
measurements should be performed. The peak \ch{NO_x} values during our
campaign lay in the region of \SI{4000}{ppb}, this is also the region
of the concentration of the injected ozone. Thus additional
measurements should be performed to determine the \ch{NO} to \ch{NO2}
conversion ratio for these high concentrations. The effect of an
increased ozone flow should be studied to see if it is a viable
solution for higher concentrations or if further oxidation processes
thwart the benefits.

All in all, the conducted experiments show that indirect \ch{NO}
measurements are possible using this setup. For moderate
concentrations reliable \ch{NO_x} values can be expected and with
slight adaptions even high ranging \ch{NO_x} values should be
determinable with a good accuracy. Thus in the near future the
converter should be ready for \ch{NO_x} vehicle measurements. There is
still some work to do when it comes to measuring \ch{NO2} and
\ch{NO_x} alternatingly. In this case further studies towards
adsorption effects are necessary and the design of the converter,
especially the reaction volume, has to be reevaluated and
adapted. Still, this work succeeded in taking a next step a robust
implementation of this promising measurement technique.

%%% Local Variables:
%%% mode: latex
%%% TeX-master: "../Bachelor"
%%% End:
