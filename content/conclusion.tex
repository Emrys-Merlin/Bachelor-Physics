\section{Conclusion and outlook}
\label{sec:conclusion}

This bachelor thesis was dedicated to further improve a \ch{NO} to
\ch{NO2} converter by testing \ch{NO_x} filtering capabilities of silica gel
behind the ozone generator. This was successfully proven. The
additional \ch{NO2} signal at the generator output could be removed
completely, while still achieving over \SI{200}{ppm} ozone
output. Moreover, the startup time seems manageable and lies in the
region of \SI{5}{\minute} after the start of the mercury lamp and the
pump of the ozone system. 

Furthermore, I could show that in a synthetic setting without \ch{NO2}
and precisely controlled \ch{NO} the measurement deviation from the
computed \ch{NO} concentrations lies in the per mille region. Hence,
there is a very high correlation between the measured and the computed
data. For measurements in ambient air it was necessary to research the
behavior of the system after turning the converter alternating.
Thereby drawbacks were investigated that \ch{NO} is still partly
converted to \ch{NO2} for \SI{2.5}{\minute} to \SI{3}{\minute} after
switching off the converter. This effect most likely arises from ozone
adsorption on the teflon walls which is only slowly removed after
turning off the converter. This has the effect that in a measurement
mode switching on and off the converter within approximately
\SI{1}{\minute} leads to an offset in the \ch{NO2} data.  The
adsorption effect is also supported by the measurement that the ozone
concentration seemed to indicate a proportionality to the reaction
tube length. If adsorption really is the reason, this would impact the
future design constraints of the converter. So far a standard
\SI{4}{\milli\meter} diameter teflon tube was used as reaction path,
which allowed for the necessary mixing of the gases and which could
easily be adapted to the necessary length of \SI{10}{\meter}. In order
to diminish the adsorption effects, the reaction path would have to be
shortened, while still keeping the dwell time constant. This could be
achieved by using a reaction tube with a larger diameter, as this
would reduce the surface area. However, a good mixing of the ozone
with the sample air still has to be guaranteed. Since the used pumps
only allow for laminar flows in the system, it might be the case that
the diameter becomes too large for an adequate mixing by diffusion. A
remedy could be the introduction of one or multiple jets behind the
ozone injection. This would allow for turbulences and improved mixing,
without unnessecarily increasing the teflon surface. Alternatively, a
shorter, thin mixing tube could be used, which then widens into a
reaction chamber. The adsorption research and possible alternative
designs of the reaction path are the natural next steps in the
development of the converter.

However, the adsorption effects have only to be taken into account, if
one wants to switch between \ch{NO_x} and \ch{NO2} measurements. If
one (or both) of the modes is (are) fixed, there are no discernible
adsorption effects. Since during vehicle measurements the actually
important quantity is the \ch{NO_x} concentration, the converter in
the current form is already ready for operation. Furthermore,
comparing the cavity data to the official data of the Heidelberg air
quality measurement station showed that using two measurement cells in
parallel avoids the difficulties introduced by the alternating
measurement mode.

The investigation of vehicel emissions showed that a true \ch{NO_x}
measurement instrument is necessary, as the average emission of
\ch{NO_x} per \ch{CO2} often lay a factor ten higher than the emission
of \ch{NO2} per \ch{CO2}. For a reliable conclusion of these emissions
a few more measurements are required. The peak \ch{NO_x} values
during the campaign lay in the region of \SI{4000}{ppb}, this is also
the region of the concentration of the injected ozone. Thus additional
measurements should be conducted to determine the \ch{NO} to \ch{NO2}
conversion ratio for these high concentrations. The effect of an
increased ozone flow should be studied, in order to see if it is a viable
solution for higher concentrations or if further oxidation processes
may arise.

All in all, in this thesis it was shown that indirect \ch{NO}
measurements are possible using this setup. For moderate
concentrations reliable \ch{NO_x} values can be achieved and with
slight adaptions even high ranging \ch{NO_x} values should be
determinable with a good accuracy. Thus in the near future the
converter should be ready for \ch{NO_x} vehicle measurements. There is
still some work to be done when one measurement cell should measure
\ch{NO2} and \ch{NO_x} alternatingly. In this case further studies
towards adsorption effects are necessary and the design of the
converter, especially the reaction volume, has to be reevaluated and
adapted. Still, this work succeeded in taking a next step towards a
robust implementation of this promising measurement technique.

%%% Local Variables:
%%% mode: latex
%%% TeX-master: "../Bachelor"
%%% End:
