\section{Introduction}
\label{sec:intro}

This bachelor thesis stands in a row of theses all concerned with the
improvement of nitrogen oxide detection capabilities of nitrogen
dioxide CE-DOAS instruments. Nitrogen oxides (\ch{NO_x}) play an
important role in the atmospherical chemistry of urban areas
(c.\,f.~\cite{roedel}) and are major air pollutants in German cities
(c.\,f.~\cite{no2schadstoff,who}). Together with other volatile
organic compounds, they can build up ozone, leading to higher health
risks. In addition, they are toxic in their own right.

In inhabitated areas, most nitrogen oxide is anthropogenic, as it is
generated during combustion processes, e.\,g.\ in car engines. Strict
regulation and monitoring by government agencies is necessary to
protect the air quality. However, the recent VW affair shows that
there is still a lack of adequate monitoring instruments. There is a
need for mobile and reasonably priced measurement units, which can be
used to determine the \ch{NO_x} emissions under live conditions; in
the streets, during regular traffic.

A first succesful step towards such an instrument is the compact
\ch{NO2} ICAD (Iterative Cavity DOAS)
instrument developed at the Institute of Environmental Physics at
Heidelberg University. It is a further development of the CE-DOAS
technique. With a led in the blue light spectrum, it can determine
nitrogen dioxide (\ch{NO2}) concentrations with a very high accuracy,
while being portable and easily installable in vehicles for in vivo
measurements. However, there is still room for improvement. Nitrogen
monoxide (\ch{NO}) is also a central component in the atmospheric
\ch{NO_x} equilibrium, but it can only be detected in the deep
ultraviolet wavelength range. This is a dilemma as UV leds are still
expensive and highly reflective mirrors in that range are so far not
available, making such an upgrade unattractive. Luckily, there is a
workaround for this problem. There is a promising conversion from
\ch{NO} to \ch{NO2}, which would allow us to measure the concentration
indirectly. The details of this converter construction are the topic
of this work.

Zimmermann~\cite{zimmermann}, Gerick~\cite{gerick} and
Jegminat~\cite{bsc} have already established, that ambient air
together with a mercury lamp can be used to generate ozone, which in
turn triggers the desired conversion. However, there are still a few
stumbling blocks left. The ozone generator produces \ch{NO2}, too,
which disturbs the measured \ch{NO_x} signals. Jegminat found that
this additional signal is due to laughing gas (\ch{N2O}), which is
very hard to filter. Therefore, this work explores the possibilities
of filtering the \ch{NO2} behind the generator, while avoiding the
removal of the necessary ozone. Silica gel is used
as~\cite{ozone-silica} inidicates, that it will quickly be saturated
by ozone, while this work investigates if it then still may adsorb
nitrogen dioxide. This work confirmed this hypothesis and determined
the filtering efffects. After having assured a clean ozone air stream,
precisely controlled \ch{NO} concentrations within synthetic air were
used to determine the accuracy of our conversion. Afterwards, the
alternating measurement mode, which switches between \ch{NO2} and
\ch{NO}+\ch{NO2} measurements, was characterized. It was found that
necessary purge times in between measurements had to be determined to
ensure that stable equilibria were reached. This investigation found
that ozone adsorption at the teflon tubes might be a major drawback,
when it comes to quickly draining the system of it. If one measurement
mode (i.\,e. \ch{NO2} or \ch{NO_x}) is fixed, no additional adsorption
effects seem to occur. With this configuration measurements using
ambient air were performed and results were compared to a
chemiluminescence monitor. Finally, this work applied the new system
to the above mentioned vehicle measurements (within Heidelberg). The
updated ICAD instrument together with a second (pure \ch{NO2})
measurement cell was installed in a car and the plume of ca.~30
vehicles was measured. Additionally, measurements next to the
Heidelberg air quality measurement station were performed, which
allowed the comparison of the ICAD results to the official data of
this station.

In the following I will first discuss the necessary physical and
chemical background for the understanding of the converter and the
DOAS system (Sec.~\ref{sec:theory}). This will already show some of
the necessary constraints for our setup, which I will describe in
detail in Section~\ref{sec:setup}. Finally, I will discuss all
performed experiments in Section~\ref{sec:measurements} together with
their results and their interpretation.

%%% Local Variables:
%%% mode: latex
%%% TeX-master: "../Bachelor"
%%% End:
