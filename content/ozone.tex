\subsection{Startup behaviour of the Ozone generator}
\label{sec:ozone}

These are the first experiments conducted using the setup described in
Section~\ref{sec:ozone-setup}. The first goal was to see whether our
setup produced any Ozone at all, i.\,e.\ whether the Silica gel became
permeable for Ozone or not. In a second step we researched the startup
behaviour of our system, as it is important to know the startup time
necessary before our generator reaches a stable Ozone output. Lastly
we checked the influence of the Pen-Ray lamp power on the Ozone
concentration. Using this setup it was impossible to test the
filtering effects on \ch{NO2} produced by the generator. Since the
concentration should be around the \ch{NO_x} concentration in ambient
lab air, it should be no more than a few \si{ppb}. This is far below
the detection limit of a longpath DOAS instrument with a pathlength of
\SI{10}{\centi\meter}. 

Turning on the generator very quickly produced an Ozone signal. Thus
we started a more in depth analysis. Turning of the generator for
about a week and then taking measurements right after turning it on,
we yielded Figure~\ref{fig:long-stop}. As can be seen the generator
takes about \SI{35}{\minute} before it reaches its stable plateau of
around \SI{250}{ppb} Ozone. During this experiment the flow was
constantly set to $\Phi_{\ch{O_3}} = \SI{0.03}{\liter\per\minute}$. It
stands to argue that the startup time could be shortened by increasing
the flow and returning to a lower value later, when the actual \ch{NO_x}
measurements are performed. Furthermore the Hg-lamp power supply was
set to \SI{10}{\milli\ampere}.

\begin{figure}[htbp]
  \centering
  \input{images/startup}
  \caption{Evolution of Ozone after a long full stop of the
    generator.}
  \label{fig:long-stop}
\end{figure}

Next we were interested in the short term startup behaviour of the
system. We stopped the generator after it had reached a stable level
and restarted it after a fixed time period. The result can be found in
Figure~\ref{fig:multiple-stop} Left. Since the time resolution of the DOAS
instrument is rather coarse the only safe statement is, that even
after a \SI{4}{\hour} stop the concentration climbed back up to around
\SI{200}{ppb} after \SI{5}{\minute}. So it stands to reason, that if
the device is used regularly the a prolonged startup time should not
be an issue, even if one sticks to the constant flow of
$\Phi_{\ch{O3}} = \SI{0.03}{\liter\per\minute}$. As compared to the
Ozone level in Figure~\ref{fig:long-stop}, the plateau seems to lie
slightly lower in this second experiment. The reason for this lies
most probably in outer conditions such as temperature in the
laboratory, which changed rather drastically from day to day, as the
experiments were conducted in November and December.

\begin{figure}[htbp]
  \centering
  % GNUPLOT: LaTeX picture with Postscript
\begingroup
  \makeatletter
  \providecommand\color[2][]{%
    \GenericError{(gnuplot) \space\space\space\@spaces}{%
      Package color not loaded in conjunction with
      terminal option `colourtext'%
    }{See the gnuplot documentation for explanation.%
    }{Either use 'blacktext' in gnuplot or load the package
      color.sty in LaTeX.}%
    \renewcommand\color[2][]{}%
  }%
  \providecommand\includegraphics[2][]{%
    \GenericError{(gnuplot) \space\space\space\@spaces}{%
      Package graphicx or graphics not loaded%
    }{See the gnuplot documentation for explanation.%
    }{The gnuplot epslatex terminal needs graphicx.sty or graphics.sty.}%
    \renewcommand\includegraphics[2][]{}%
  }%
  \providecommand\rotatebox[2]{#2}%
  \@ifundefined{ifGPcolor}{%
    \newif\ifGPcolor
    \GPcolorfalse
  }{}%
  \@ifundefined{ifGPblacktext}{%
    \newif\ifGPblacktext
    \GPblacktexttrue
  }{}%
  % define a \g@addto@macro without @ in the name:
  \let\gplgaddtomacro\g@addto@macro
  % define empty templates for all commands taking text:
  \gdef\gplbacktext{}%
  \gdef\gplfronttext{}%
  \makeatother
  \ifGPblacktext
    % no textcolor at all
    \def\colorrgb#1{}%
    \def\colorgray#1{}%
  \else
    % gray or color?
    \ifGPcolor
      \def\colorrgb#1{\color[rgb]{#1}}%
      \def\colorgray#1{\color[gray]{#1}}%
      \expandafter\def\csname LTw\endcsname{\color{white}}%
      \expandafter\def\csname LTb\endcsname{\color{black}}%
      \expandafter\def\csname LTa\endcsname{\color{black}}%
      \expandafter\def\csname LT0\endcsname{\color[rgb]{1,0,0}}%
      \expandafter\def\csname LT1\endcsname{\color[rgb]{0,1,0}}%
      \expandafter\def\csname LT2\endcsname{\color[rgb]{0,0,1}}%
      \expandafter\def\csname LT3\endcsname{\color[rgb]{1,0,1}}%
      \expandafter\def\csname LT4\endcsname{\color[rgb]{0,1,1}}%
      \expandafter\def\csname LT5\endcsname{\color[rgb]{1,1,0}}%
      \expandafter\def\csname LT6\endcsname{\color[rgb]{0,0,0}}%
      \expandafter\def\csname LT7\endcsname{\color[rgb]{1,0.3,0}}%
      \expandafter\def\csname LT8\endcsname{\color[rgb]{0.5,0.5,0.5}}%
    \else
      % gray
      \def\colorrgb#1{\color{black}}%
      \def\colorgray#1{\color[gray]{#1}}%
      \expandafter\def\csname LTw\endcsname{\color{white}}%
      \expandafter\def\csname LTb\endcsname{\color{black}}%
      \expandafter\def\csname LTa\endcsname{\color{black}}%
      \expandafter\def\csname LT0\endcsname{\color{black}}%
      \expandafter\def\csname LT1\endcsname{\color{black}}%
      \expandafter\def\csname LT2\endcsname{\color{black}}%
      \expandafter\def\csname LT3\endcsname{\color{black}}%
      \expandafter\def\csname LT4\endcsname{\color{black}}%
      \expandafter\def\csname LT5\endcsname{\color{black}}%
      \expandafter\def\csname LT6\endcsname{\color{black}}%
      \expandafter\def\csname LT7\endcsname{\color{black}}%
      \expandafter\def\csname LT8\endcsname{\color{black}}%
    \fi
  \fi
    \setlength{\unitlength}{0.0500bp}%
    \ifx\gptboxheight\undefined%
      \newlength{\gptboxheight}%
      \newlength{\gptboxwidth}%
      \newsavebox{\gptboxtext}%
    \fi%
    \setlength{\fboxrule}{0.5pt}%
    \setlength{\fboxsep}{1pt}%
\begin{picture}(3888.00,3888.00)%
    \gplgaddtomacro\gplbacktext{%
      \csname LTb\endcsname%
      \put(814,704){\makebox(0,0)[r]{\strut{}$80$}}%
      \put(814,1121){\makebox(0,0)[r]{\strut{}$100$}}%
      \put(814,1538){\makebox(0,0)[r]{\strut{}$120$}}%
      \put(814,1955){\makebox(0,0)[r]{\strut{}$140$}}%
      \put(814,2372){\makebox(0,0)[r]{\strut{}$160$}}%
      \put(814,2789){\makebox(0,0)[r]{\strut{}$180$}}%
      \put(814,3206){\makebox(0,0)[r]{\strut{}$200$}}%
      \put(814,3623){\makebox(0,0)[r]{\strut{}$220$}}%
      \put(946,484){\makebox(0,0){\strut{}$0$}}%
      \put(1455,484){\makebox(0,0){\strut{}$2$}}%
      \put(1964,484){\makebox(0,0){\strut{}$4$}}%
      \put(2473,484){\makebox(0,0){\strut{}$6$}}%
      \put(2982,484){\makebox(0,0){\strut{}$8$}}%
      \put(3491,484){\makebox(0,0){\strut{}$10$}}%
    }%
    \gplgaddtomacro\gplfronttext{%
      \csname LTb\endcsname%
      \put(176,2163){\rotatebox{-270}{\makebox(0,0){\strut{}Concentration [ppm]}}}%
      \put(2218,154){\makebox(0,0){\strut{}Time [min]}}%
      \csname LTb\endcsname%
      \put(2504,1537){\makebox(0,0){\nfrac{} 1/2 h}}%
      \csname LTb\endcsname%
      \put(2504,1317){\makebox(0,0){1 h}}%
      \csname LTb\endcsname%
      \put(2504,1097){\makebox(0,0){2 h}}%
      \csname LTb\endcsname%
      \put(2504,877){\makebox(0,0){4 h}}%
    }%
    \gplbacktext
    \put(0,0){\includegraphics{../images/multi}}%
    \gplfronttext
  \end{picture}%
\endgroup

  \hfill
  % GNUPLOT: LaTeX picture with Postscript
\begingroup
  \makeatletter
  \providecommand\color[2][]{%
    \GenericError{(gnuplot) \space\space\space\@spaces}{%
      Package color not loaded in conjunction with
      terminal option `colourtext'%
    }{See the gnuplot documentation for explanation.%
    }{Either use 'blacktext' in gnuplot or load the package
      color.sty in LaTeX.}%
    \renewcommand\color[2][]{}%
  }%
  \providecommand\includegraphics[2][]{%
    \GenericError{(gnuplot) \space\space\space\@spaces}{%
      Package graphicx or graphics not loaded%
    }{See the gnuplot documentation for explanation.%
    }{The gnuplot epslatex terminal needs graphicx.sty or graphics.sty.}%
    \renewcommand\includegraphics[2][]{}%
  }%
  \providecommand\rotatebox[2]{#2}%
  \@ifundefined{ifGPcolor}{%
    \newif\ifGPcolor
    \GPcolorfalse
  }{}%
  \@ifundefined{ifGPblacktext}{%
    \newif\ifGPblacktext
    \GPblacktexttrue
  }{}%
  % define a \g@addto@macro without @ in the name:
  \let\gplgaddtomacro\g@addto@macro
  % define empty templates for all commands taking text:
  \gdef\gplbacktext{}%
  \gdef\gplfronttext{}%
  \makeatother
  \ifGPblacktext
    % no textcolor at all
    \def\colorrgb#1{}%
    \def\colorgray#1{}%
  \else
    % gray or color?
    \ifGPcolor
      \def\colorrgb#1{\color[rgb]{#1}}%
      \def\colorgray#1{\color[gray]{#1}}%
      \expandafter\def\csname LTw\endcsname{\color{white}}%
      \expandafter\def\csname LTb\endcsname{\color{black}}%
      \expandafter\def\csname LTa\endcsname{\color{black}}%
      \expandafter\def\csname LT0\endcsname{\color[rgb]{1,0,0}}%
      \expandafter\def\csname LT1\endcsname{\color[rgb]{0,1,0}}%
      \expandafter\def\csname LT2\endcsname{\color[rgb]{0,0,1}}%
      \expandafter\def\csname LT3\endcsname{\color[rgb]{1,0,1}}%
      \expandafter\def\csname LT4\endcsname{\color[rgb]{0,1,1}}%
      \expandafter\def\csname LT5\endcsname{\color[rgb]{1,1,0}}%
      \expandafter\def\csname LT6\endcsname{\color[rgb]{0,0,0}}%
      \expandafter\def\csname LT7\endcsname{\color[rgb]{1,0.3,0}}%
      \expandafter\def\csname LT8\endcsname{\color[rgb]{0.5,0.5,0.5}}%
    \else
      % gray
      \def\colorrgb#1{\color{black}}%
      \def\colorgray#1{\color[gray]{#1}}%
      \expandafter\def\csname LTw\endcsname{\color{white}}%
      \expandafter\def\csname LTb\endcsname{\color{black}}%
      \expandafter\def\csname LTa\endcsname{\color{black}}%
      \expandafter\def\csname LT0\endcsname{\color{black}}%
      \expandafter\def\csname LT1\endcsname{\color{black}}%
      \expandafter\def\csname LT2\endcsname{\color{black}}%
      \expandafter\def\csname LT3\endcsname{\color{black}}%
      \expandafter\def\csname LT4\endcsname{\color{black}}%
      \expandafter\def\csname LT5\endcsname{\color{black}}%
      \expandafter\def\csname LT6\endcsname{\color{black}}%
      \expandafter\def\csname LT7\endcsname{\color{black}}%
      \expandafter\def\csname LT8\endcsname{\color{black}}%
    \fi
  \fi
    \setlength{\unitlength}{0.0500bp}%
    \ifx\gptboxheight\undefined%
      \newlength{\gptboxheight}%
      \newlength{\gptboxwidth}%
      \newsavebox{\gptboxtext}%
    \fi%
    \setlength{\fboxrule}{0.5pt}%
    \setlength{\fboxsep}{1pt}%
\begin{picture}(3888.00,3888.00)%
    \gplgaddtomacro\gplbacktext{%
      \csname LTb\endcsname%
      \put(814,704){\makebox(0,0)[r]{\strut{}$200$}}%
      \put(814,996){\makebox(0,0)[r]{\strut{}$220$}}%
      \put(814,1288){\makebox(0,0)[r]{\strut{}$240$}}%
      \put(814,1580){\makebox(0,0)[r]{\strut{}$260$}}%
      \put(814,1872){\makebox(0,0)[r]{\strut{}$280$}}%
      \put(814,2164){\makebox(0,0)[r]{\strut{}$300$}}%
      \put(814,2455){\makebox(0,0)[r]{\strut{}$320$}}%
      \put(814,2747){\makebox(0,0)[r]{\strut{}$340$}}%
      \put(814,3039){\makebox(0,0)[r]{\strut{}$360$}}%
      \put(814,3331){\makebox(0,0)[r]{\strut{}$380$}}%
      \put(814,3623){\makebox(0,0)[r]{\strut{}$400$}}%
      \put(946,484){\makebox(0,0){\strut{}$0$}}%
      \put(1229,484){\makebox(0,0){\strut{}$5$}}%
      \put(1512,484){\makebox(0,0){\strut{}$10$}}%
      \put(1794,484){\makebox(0,0){\strut{}$15$}}%
      \put(2077,484){\makebox(0,0){\strut{}$20$}}%
      \put(2360,484){\makebox(0,0){\strut{}$25$}}%
      \put(2643,484){\makebox(0,0){\strut{}$30$}}%
      \put(2925,484){\makebox(0,0){\strut{}$35$}}%
      \put(3208,484){\makebox(0,0){\strut{}$40$}}%
      \put(3491,484){\makebox(0,0){\strut{}$45$}}%
    }%
    \gplgaddtomacro\gplfronttext{%
      \csname LTb\endcsname%
      \put(176,2163){\rotatebox{-270}{\makebox(0,0){\strut{}Concentration [ppm]}}}%
      \put(2218,154){\makebox(0,0){\strut{}Time [min]}}%
    }%
    \gplbacktext
    \put(0,0){\includegraphics{../images/current}}%
    \gplfronttext
  \end{picture}%
\endgroup

  \caption{Left: Evolution of the Ozone concentration after a full stop of the
    generator. Right: Ozone level dependence on current of Hg-lamp. The steep
    flank occured after a change of the current from
    \SI{10}{\milli\ampere} to \SI{17}{\milli\ampere}.}
  \label{fig:multiple-stop}
\end{figure}

After having analyzed the startup procedure, we wanted to at least get
the qualitative influence of the power supply current of the Pen-Ray
lamp on the Ozone concentration. Beforehand it was not clear whether a
higher current would increase or decrease the Ozone production rate,
as, as was described in Section~\ref{sec:theory-ozone}, we did not
know how a change in power would transform the power distribution on
the different Mercury lines. We recorded a timeseries while switching
between the two possible currents, \SI{10}{\milli\ampere} and
\SI{17}{\milli\ampere}, and yielded 
Figure~\ref{fig:multiple-stop} Right. From this we see that we still work in a regime
where higher power leads to more Ozone. This is interesting to know,
however, since \SI{200}{ppb} is already more than enough Ozone for our
purposes, in the practical application we just used the current
which was easiest to set up\footnote{The currents of all used supplies
  lay in the same region.}.

%%% Local Variables: 
%%% mode: latex
%%% TeX-master: "../Bachelor"
%%% End: 
