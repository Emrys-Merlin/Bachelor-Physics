\section{Ozone production}
\label{sec:ozone}

\subsection{Ozone generator setting}
\label{sec:ozone-setting}



\subsection{Measurement}
\label{sec:ozone-meas}

It is necessary to understand the characterstic of our instrument
before we use it to convert \ch{NO} to \ch{NO2}. For that we measured
the time it took to reach a stable production of Ozone and where this
level lay. 

In figure~\ref{fig:long-stop} one can see the time series of the Ozone
level of the generator, if the generator had been turned of for a long
time\footnote{In this case about a week.}. As can be seen the level
rises rather quickly and reaches a stable plateau after approximately
\SI{35}{\minute}. The equilibrium lies around 250ppm.

\begin{figure}[htbp]
  \centering
  \begin{tikzpicture}
    \begin{axis}[
      xlabel={Time [\si{\hour}]},
      ylabel={c(\ch{O3}) [ppb]},
      no markers,
      xmin=0,
      xmax=2,
      ]
      \addplot table [col sep=comma] {images/Anschaltkurve2.csv};
    \end{axis}
  \end{tikzpicture}
  \caption{Evolution of Ozone after a long full stop of the
    generator.}
  \label{fig:long-stop}
\end{figure}

In a next step we determined the time necessary to get back to the
equilibrium after shorter stops of the generator. For this we measured
the time series of the Ozone concentration after a stop of {\nfrac
  1/2} \si{\hour}, \SI{1}{\hour}, \SI{2}{\hour} and \SI{4}{\hour}. The
result can be found in figure~\ref{fig:multiple-stop}, which shows
that the Ozone reaches the saturation level within
\SI{10}{\minute}. That in this case the saturation level lay around
200 ppm can be explained by the difference in temperature between the
two measurements. 

\begin{figure}[htbp]
  \centering
  \begin{tikzpicture}
    \begin{axis}[
      xlabel={Time [\si{\minute}]},
      ylabel={c(\ch{O3}) [ppb]},
      legend entries={\nfrac 1/2\si{\hour}, \SI{1}{\hour}, \SI{2}{\hour},
        \SI{4}{\hour}},
      no markers,
      legend pos=south east,
      xmin=0,
      xmax=10,
      ]
      \addplot table [col sep=comma] {images/Multiple-stop.csv};
      \addplot table [col sep=comma,y=1h] {images/Multiple-stop.csv};
      \addplot table [col sep=comma,y=2h] {images/Multiple-stop.csv};
      \addplot table [col sep=comma,y=4h] {images/Multiple-stop.csv};
    \end{axis}
  \end{tikzpicture}
  \caption{Evolution of the Ozone concentration after a full stop of the
    generator.}
  \label{fig:multiple-stop}
\end{figure}

In a last step we researched the influence of a higher current at the
Penray lam on the Ozone level. Figure~\ref{fig:lamp} shows that after
turning on the lamp the saturation level rises from 200 ppm to over
350 ppm.

\begin{figure}[htbp]
  \centering
  \begin{tikzpicture}
    \begin{axis}[
      xlabel={Time [\si{\minute}]},
      ylabel={c(\ch{O3}) [ppb]},
      no markers,
      xmin=0,
      xmax=45,
      ]
      \addplot table [col sep=comma] {images/2015-12-02-current2.csv};
    \end{axis}
  \end{tikzpicture}
  \caption{Ozone level dependence on current of Hg-lamp.}
  \label{fig:lamp}
\end{figure}

%%% Local Variables: 
%%% mode: latex
%%% TeX-master: "../Bachelor"
%%% End: 
