\subsection{Onset after Ozone switch}
\label{sec:switch}

This measurement was a preliminary measurement to perform the 
transition from our pure Nitrogen Monoxide measurement to unfiltered
ambient air. The new challenge consists in the obvious fact, that
ambient air is not \ch{NO2} free. Thus if we add Ozone to our sample
air stream, we measure \ch{NO_x\, =\ NO + NO2}. In some cases, e.\,g.\
during live vehicle measurements this is exactly what we want
(c.\,f.~Sec.~\ref{sec:vehicle}), however, in other cases we would like
to obtain the Nitrogen Monoxide and the Nitrogen Dioxide
concentrations separately. This can only be achieved indirectly. If
we have only one cavity, we can choose to measure \ch{NO2} and
\ch{NO_x} alternatingly by switching the Ozone stream off and
on. Doing that the technical question arises how long we have to wait
after triggering an Ozone switch, before the equilibrium is reached
and we can take our next measurement.

As Section~\ref{sec:requirements} indicates, for a pathlength of
\SI{15}{\meter} we need about \SI{5.3}{\second} for the gas to be
completely exchanged. Taking the cavity itself into account a
\SI{30}{\second} purge time should well suffice for the exchange. As
will be seen during this section, there are other effects, which have
a far longer decay time.

\subsubsection{Setup}
\label{sec:switch-setup}

This experiment was performed using the same setup as in
Section~\ref{sec:no} (c.\,f.\ Fig.~\ref{fig:no-setup} on
page~\pageref{fig:no-setup}). We fixed the flow of the \ch{NO}
calibration gas to $\Phi_{\ch{NO}} = \SI{0.01}{\liter\per\minute}$ and
kept all other flows as before. We set the cavity measurement script
to record 60 spectra sampled over 1000 scans each with an exposure
time of \SI{10}{\milli\second} then switch the Ozone stream on, record
60 spectra with the same characteristcs, switch the Ozone stream off
and repeat. With this setup we got a time resolution of
\SI{10}{\second}. In hindsight the sample size could have been
smaller to improve the time resolution. We performed the measurement
for both \SI{5}{\meter} and \SI{15}{\meter} reaction pathlength, due
to time constraints the measurement for \SI{10}{\meter} was shortened
to only collect 30 spectra. Thus we are missing some of the decay tail
in this case.

\subsubsection{Results}
\label{sec:switch-results}

Figure~\ref{fig:switch} contains the timeseries of the measured
\ch{NO2} concentration for a reaction pathlength of $l =
\SI{15}{\meter}$. The timeseries for the other two pathlengths were
similar in shape and were therefore omitted. The flanks correspond to
the switch in the Ozone flow. There are severeal intersting
observations concerning this plot. First of all the flanks upwards are
very steep, which in this setting means, that the transition takes
less than \SI{10}{\second}. The decreasing flanks take longer and
interestingly they do not reach \SI{0}{ppb}. Additionally right after
the start of the measurement (before any Ozone had been added to the
sample air stream) the concentration rose to a level of ca.\
SI{1.3}{ppb}. One explanation for this is, that the calibration gas is
not \ch{NO2} free. This would raise the question why we did not
measure too much \ch{NO} in Section~\ref{sec:no}, since we actually
measured \ch{NO_x}. This question has two possible answers. First, the
\ch{NO} could have converted to \ch{NO2} over time in a ratio of
1:1. This would explain why we got the right results with the official
container concentration. Alternatively, the official concentration
might not have distinguished between \ch{NO} and \ch{NO2}, but might
have only been talking about a \ch{NO_x} concentration. This second
idea could be rejected after discovering the calibration method, which
indeed distinguished between \ch{NO} and \ch{NO2}.

Accepting for the moment that we have a background \ch{NO2} signal, we
turn towards the question, why the decreasing flanks have the slowly
falling tail. In Figure~\ref{fig:switch-pl}, there falling flanks for
different pathlenghts and we see that they all appear to have a
similar shape. They all seem to tend to the same equilibrium and the
decay time seems to grow with the pathlength. We suspect the reason
for this long tail to be a technical one. Between the Ozone valve and
the t-piece, connecting th Ozone stream to the sample air stream,
there is about \SI{14}{\centi\meter} tube filled with around
\SI{200}{ppm} Ozone. If this diffused into the sample line, this would
lead to further \ch{NO} conversion, although the Ozone stream was
already switched off. To investigate this hypothesis, we tried to fit
the decay using the following fit function
\begin{align}
  f(t) \coloneqq c_0 + c_f \cdot\exp\left( -\frac{t}{\tau_f}\right) +
  c_s \cdot \exp\left(-\frac{t}{\tau_s}\right), \label{eq:switch-fit}
\end{align}
i.\,e.\ we suspect two overlaying decay processes. A fast one with a
decay time of about $\tau_f \approx \SI{10}{\second}$ and a slower one
explaining the tail. Fitting this function we yield the values
summarized in Table~\ref{tab:switch-coeff}. For the \SI{10}{\meter}
pathlength fit the offset concentration had to be fixed, as the
measured tail was too short for it to be determined correctly. 

\begin{table}[hbtp]
  \centering
  \begin{tabular}{rSSSSSS}
    \toprule
    {$l$}& {$\tau_f$} & {$\tau_s$} & {$c_0$} & {$c_f$} & {$c_s$} & {$V_{\ch{O3}}$}\\
    {\si{\meter}}& {\si{\second}} & {\si{\second}} & {\si{ppb}} & {\si{ppb}} &
                                                      {\si{ppb}} & {\si{10\tothe{-6}\milli\liter}}\\
    \midrule
    5 & 5.95 \pm 0.13 & 141 \pm 14 & 1.48 \pm 0.03 & 16.5 \pm 0.1 
                      & 1.46 \pm 0.07 & 6.9 \pm 0.8\\
    10 & 7.04 \pm 0.15 & 146 \pm 10 & 1.5 & 19.6 \pm 0.2
                       & 2.1 \pm 0.1 & 10.2 \pm 0.1\\
    15 & 10.15 \pm 0.12 & 193 \pm 12 & 1.52 \pm 0.03 & 21.0 \pm 0.1
                        & 2.36 \pm 0.06 & 15.2 \pm 0.2\\
    \bottomrule
  \end{tabular}
  \caption{Fit coefficients for the decay function
    (Eq.~\eqref{eq:switch-fit}) after an Ozone switch off. For the
    pathlength of $l= \SI{10}{\meter}$ the offset concentration was
    fixed to \SI{1.5}{ppb}. This was necessary as, due to the
    shortness of the measurement time, the tail was not long enough
    for the fit to determine the offset correctly. The last column
    contains the (partial) Volume of the Ozone participating in the
    reaction to form the long tail.}
  \label{tab:switch-coeff}
\end{table}
In any case the fit proves that we work with two time scales. First a
rather fast decay, which is well described by the time necessary for
the gas to purge the system. And second a time scale in on the order
from \SIrange{2.5}{3}{\minute}. To test our hypothesis we want to use
the fits to determine the amount of Ozone necessary to explain the
tail. We assume that the measured \ch{NO2} signal above the offset,
comes directly from the conversion of \ch{NO} to \ch{NO2}. Looking at
Reaction~\eqref{eq:no}, we see that this relation leads to a one to
one correspondence between the \ch{NO2} signal\footnote{always above
  the offset} and the amount of Ozone enetering the sample air
stream. To determine the absolute amount, we integrate the slow
falling part of the fit function and use the constant flow of $\Phi =
\SI{2}{\liter\per\minute}$ to convert the result to the (partial)
volume of Ozone, i.\,e.
\begin{align*}
  V_{\ch{O3}} \coloneqq  c_s \cdot \tau_s \cdot \Phi.
\end{align*}
The result of this computation can be found in
Table~\ref{tab:switch-coeff}, too. We first notice the clear
pathlength dependence. This is already an indicator against our
hypothesis, as our dead volume is constant. Comparing it still to the
Ozone volume in the tube we get
\begin{align*}
  V_{\text{tube}} \approx \SI{3.52e-4}{\milli\liter(Ozone)}.
\end{align*}
This is even more evidence that the dead volume cannot be the sole
explanation. The total Ozone volume is about two orders of magnitude
too large. Even if we take into account that probably not the whole
tube volume contributes to the reaction (as the Ozone enters the
sample air stream via diffusion), we still have to explain the
pathlength dependence. Especially as it seems that the relation is
highly linear, one might suspect the reason to lie in the adsorption
capacity of the teflon tubes. The adsorption capacity for Ozone would
have to be researched more in depth than we can provide at the current
time, however, if the adsorptoin plays an important role in the decay
behaviour of the system, then we have a further important restriction
concerning the measurement method. If would become basically
impossible to use one cavity to determine \ch{NO_x} and \ch{NO2}
alternatingly, as the purge time would have to be in the order of
magnitude of multiple minutes. This is inacceptable for volatile
environments as for example urban areas.

\begin{figure}[htbp]
  \centering
  \input{images/20160223_15_equil_fixI0_ts}
  \caption{Timeseries of the measured \ch{NO2} concentration while
    switching the Ozone stream on and off.}
  \label{fig:switch}
\end{figure}
Lastly, we turn towards possibilities to account for the effects of
the long tail, if the purge time was chosen too short. In these
laboratory conditions, we know that we had a constant supply of
\ch{NO}, such that the behaviour of the falling flank is solely
determined by the changing Ozone concentration. If we look at a
special, but in this work often used case, we are interested in the
setting where we have \SI{30}{\second} purge time and afterwards
\SI{30}{\second} of measurement (300 spectra à
\SI{10}{\milli\second}). Then we can compute the average additional
\ch{NO2} signal coming from the slow tail

\begin{align*}
  \bar c_s = \frac{c_s}{\Delta t} \int_{t_0}^{t_1}
  \exp\left(\frac{t}{\tau_s}\right)\d[t] \approx \SI{1.55}{ppb},
\end{align*}
with $t_1 = \SI{30}{\second}$, $t_2 = \SI{60}{\second}$, $\Delta t =
t_2 - t_1$ and specialising to the case of pathlength $l =
\SI{10}{\meter}$. We propose that this value is mostly independent of
the precise \ch{NO} concentration present in the system, as long as it
is larger than $c_{\text{crit}} \approx \SI{1.7}{ppb}$. Then we expect
the bottle neckt of the Reaction~\eqref{eq:no} to be the diminishing
\ch{O3} concentration, which we assume to be well described by the
above decay plots. If these assumptions are correct and we are in a
region with nonvanishing \ch{NO} concentration, then we should be able
to mitigate the effects of the tail, by subtracting the above computed
average slow tail concentration $\bar c_s$ from the measured \ch{NO2}
concentrations. The results of this procedure can be seen in the next
section.
\begin{figure}[htbp]
  \centering
  \input{images/20160223_equil_fixI0_fall_fit}
  \caption{The behaviour of the falling flanks depending on the
    reaction pathlength}
  \label{fig:switch-pl}
\end{figure}


%%% Local Variables:
%%% mode: latex
%%% TeX-master: "../Bachelor"
%%% End:
