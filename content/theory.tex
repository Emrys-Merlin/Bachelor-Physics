\section{Theoretical background}
\label{sec:theory}

\todo{reference old bachelor thesis}

\subsection{The rate coefficient}
\label{sec:rate}

\todo{write something about it}


\subsection{Ozone Generation}
\label{sec:theory-ozone}

In this section we will look into a possible way of generating Ozone
(\ch{O3}) out of ambient air. This step is crucial, as the Ozone is
central compound in the reaction that transforms \ch{NO} to
\ch{NO2}. Furthermore it is important to clear the air of laughing
gas as can be seen in section\todo{write something about it}.

The equations of the following generation cycle ar often referred to
as the \emph{Chapman-Cycle}\todo{reference}. These equations not only
describe the production but also the destruction of Ozone.

The starting point is a UV-photon which splits an Oxygen molecule:

\begin{align*}
  \ch{O2} + h\nu \ch{->[$\lambda<\SI{242}{\nano\meter}$]} \ch{2 O(^3P)}.
\end{align*}

Together with a particle M for momentum conversation the $\ch{O(^3
  P)}$ radical can produce ozone as shown in the following formula

\begin{align}
  \ch{O(^3 P) + O2 + M -> O3 + M}. \label{eq:ozone}
\end{align}

However, \ch{O3} is not stable, if there is a UV source present. If
there is light with wavelength under $\SI{310}{\nano\meter}$ then the
Ozone will be split again

\begin{align}
  \ch{O3} + h\nu \ch{->[$\lambda<\SI{310}{\nano\meter}$] O(^1 D) +
  O2}. \label{eq:split}
\end{align}

In a next step the $\ch{O(^1 D)}$ radical can be reexcited via

\begin{align*}
  \ch{O(^1 D) + M -> O(^3 P) + M}
\end{align*}

and thus can reenter into equation~\eqref{eq:ozone}. The other
possibility for a $\ch{O(^3 P)}$ radical is to either react with
another radical or an Ozone molecule which would reduce the Ozone
concentration. In formulas this would take the form

\begin{align*}
  \ch{2 O(^3 P) + M & -> O2 + M}\\
  \ch{O(^3 P) + O3 & -> 2 O2}.
\end{align*}

Thus we see that if we introduce a UV light source with wavelength
less than \SI{242}{\nano\meter}, we can generate Ozone
by splitting Oxygen. Some of it will be destroyed again by the split
in equation~\eqref{eq:split}, so at some point we will enter an
equilibrium state and thus a stable concentration of \ch{O3} that can
be used in the later application.

\subsection{Ozone and Nitrogen Oxide}
\label{sec:o-no}

There are several reactions which are triggered by an abundance of
\ch{O3}. The for our purpose desired one is the following\todo{find
  reference for rate coefficients}

\begin{align*}
  \ch{NO + O3 & ->[$k=\SI{1.8e-14}{\hertz\per\cubic\centi\meter}$] NO2 + O2}.
\end{align*}

Thus we can use the generated Ozone as an easyway to convert \ch{NO}
to \ch{NO2}. If this were the only reaction, we could use this method
to determine the \ch{NO} concentration perfectly. However, there are
more reactions taking place, which make it harder to use the \ch{NO2}
concentration to conclude the \ch{NO} concentration. For once Ozone
reacts directly with \ch{NO2}, too. We yield

\begin{align*}
  \ch{NO2 + O3 ->[$k=\SI{3.5e-17}{\hertz\per\cubic\centi\meter}$] NO3 + O2}.
\end{align*}

We see that part of the \ch{NO2} will be converted to \ch{NO3}, which
cannot be detected by our measuring instrument. Even worse, \ch{NO3}
itslef will react with \ch{NO2} and generate laughing gas, i.e.

\begin{align}
  \ch{NO2 + NO3
  <=>[$k=\SI{1.9e-12}{\hertz\per\cubic\centi\meter}$][$k=\SI{2.6e-11}{\hertz\per\cubic\centi\meter}$] N2O5}. \label{eq:laughing}
\end{align}

Luckily, the equilibrium of equation~\eqref{eq:laughing} lies on the
side of the educts, so the effect should not be too large.

One last equation that has to be taken into account is

\begin{align}
  \ch{NO + NO3 ->[$k=\SI{6.9e-2}{\hertz\per\cubic\centi\meter}$] 2 NO2}.\label{eq:back}
\end{align}

Since it has such a large rate coefficient, we see that the conversion
of \ch{NO2} to \ch{NO3} can be expected to not be that influential, as
most of it will degrade to \ch{NO2} again, as long as there is still
\ch{NO}. So in theory we should be able to adjust the amount of Ozone
mixed with the sample air in such a way that equation~\eqref{eq:back}
makes sure that none of the \ch{NO2} is lost.


\subsection{The physics behind CE-DOAS}
\label{sec:ce-doas-physics}

In the following sections we will discuss the physical concepts
necessary to understand the Cavity Enhance Diffirential Absorption
Spectroscopy (CE-DOAS). First, we will investigate the Lambert-Beer
Law. This leads directly to the so called \emph{Longpath}-DOAS
instrument. As the name already indicates, those need a long
lightpath, since normally the absorption crosssections are very
small. Thus lastly we introduce the \emph{cavity enhancement} to
overcome this restriction.

The presentation follows mostly~\cite{fp58} with a few inspirations
taken from~\cite{platt2008differential}.

\subsubsection{The Lambert-Beer Law}
\label{sec:lambert-beer}

Behind the DOAS method stands an easily understood concept, the so
called \emph{Lambert-Beer Law}. We take a narrow ligth source emitting a
wavelength dependent intensity $I_0(\lambda)$ and ask ourselves the
question how the intensity will behave in a distance $L$. If we
neglect broadening of the light beam the intensity should mainly
depend on the absorption behaviour of the matter it passes. If we can
make the further assumption that our matter is isotropic then it makes
only sense that the decrease of the intensity only depends on the
intensity itself, but not on the exact location on the path. Thus we
yield a very well known differential equation

\begin{align*}
  \frac{\partial I}{\partial L}(\lambda, L) = - \epsilon(\lambda)
  \cdot I(L, \lambda),
\end{align*}
where $\epsilon$ is called the absorption coefficient, which is of
course wave length dependent. Integrating this equation, using $I_0$
as initial value, we get the standard form of the Lambert-Beer Law

\begin{align}
  I(\lambda, L) = I_0(\lambda) \cdot \exp[-\epsilon(\lambda) \cdot
  L]. \label{eq:lb-easy}
\end{align}

This law describes the absorption behaviour of light very well for our
purposes. However, in this form it the connection to the trace gas
abundance is not completely clear, but that is what we are interested
in in the end.

The concentration dependence is hidden in the $\epsilon$. It seems
natural, that the absorption should be proportional to the
concentratoin of the matter. So we can write

\begin{align*}
  \epsilon(\lambda) = \sigma(\lambda) \cdot c,
\end{align*}

where proporitonality constant $\sigma$ is denoted \emph{absorption
  corsssection} and $c$ is the concentration. To be even more precise
we should write

\begin{align}
  \epsilon(\lambda) = \sum_{i=1}^n \sigma_i(\lambda) \cdot c_i, \label{eq:lb-abs}
\end{align}

where we have $n$ different trace gases we want to consider in the
absorption process. 

Equations~\eqref{eq:lb-easy} and~\eqref{eq:lb-abs} now contain the central idea for our
spectroscopic method. We can measure the intensity spectra $I$ and
$I_0$, we can also determine the pathlength $L$. If we now want to
compute the concentration of our trace gases, we only need the
absorption crosssections, which can be found in the
literature.\todo{cite} Thus lastly we introduce the \emph{Optical
  Density} 

\begin{align*}
  D(\lambda) = \ln \left(\frac{I(\lambda)}{I_0(\lambda)}\right) = - L
  \cdot \sum_{i=1}^n \sigma_i(\lambda) c_i,
\end{align*}

which contains all the necessary information.

\subsubsection{The DOAS method}
\label{sec:doas}

\todo{write something about units?}

Differential Optical Absorption Spectroscopy uses the Lambert-Beer Law
to compute the concentrations of trace gases in the air. There are a
few more technicalities we need to respect. First of all, we must have
a look at the absorption crosssections of different trace gases. If we
want to be able to discern different gases and even compute their
concentration, we need the crosssections to be sufficiently
different.
\todo{used in section 3.1 or something}

As can be seen in Figure\todo{include picture of crosssection}  the
crosssections can be separated into two parts: One only weakly
wavelength dependent and similar for all gases and a narrowband
structure on top which differs strongly for different species. This
narrowband structure is what will allow us to discern the gases. It is
called the \emph{differential spectrum}. Hence the name of the
method. In the actual fitting process we will first remove the
broadband structure, to be only left with the differentials.

However, there are still two effects we need to take into
account. Absorption is not the only effect that decreases the
intensity of our light beam in air. There are also scattering
processes taking place. Namely Rayleigh and Mie scattering. These
continous effects can be accounted for by adding wavelength dependent
polynomial to the fit function. Thus we fit

\begin{align*}
  f(\lambda) = - \sum_{i=1}^n c_i \cdot L \cdot \sigma_i(\lambda) +
  \sum_k a_k \lambda^k
\end{align*}

to the Optical Density, where the concentrations $c_i$ and the polynomial
coefficients $a_k$ are to be determined.

\todo{write something about instrument function}

\subsubsection{The CE-DOAS method}
\label{sec:ce-doas}

Since the absorption coefficents $\epsilon$ are so small, we need a
very long pathlength $L$ to see a signal in the DOAS method. One
possibility is to really use a very long geometrical pathlength using
mirrors and telescopes. With this we are able to obtain pathlengths
between \num{1} and \SI{10}{\kilo\meter}\todo{cite}. These so called
\emph{Longpath-DOAS} instruments have advantages, but also a few
disadvantages one of them clearly being that we need a lot of space
and in the process we average the concentration of our trace gases
over a large area.

An alternative approach is to increase the lightpath by the use of
optical instruments, i.\,e.\ mirrors. Using two highly reflective
mirrors we can obtain very large optical pathlengths, while still
having a managable geometrical pathlength. The price we pay by taking
this approach lies in the fact that now our pathlength becomes
wavelength dependent as this is true for the reflectivity of all
highly reflecting mirrors. 

In this paragraph we will generalize the DOAS approach keeping in mind
that our pathlength is not constant anymore. On the way we try to
separate two effects on the pathlength. First the shortening due to
the absorption by the trace gases and secondly the influence by the
reflectivity.

The CE-DOAS method was used for the measurements in
Sections~\ref{sec:o3-setting-doas} to~\ref{sec:vehicle}. 

The geometrical setup is the following:\todo{setup}

where we have two mirrors with reflectivity $R_i$, transmittance $T_i$
and absorption $A_i$. We have

\begin{align*}
  R_i + T_i + A_i = 1 \quad i \in{1,2}.
\end{align*}

Furthermore we have the transmittance $T_g$ of the gas in the
cavity. For this we get from the Lamert-Beer law

\begin{align*}
  T_g = \exp(-\epsilon d),
\end{align*}

where $d$ is the length of our cavity. If we take some intensity
$I_{\text{in}}$ entering the cavity and we want to compute the outgoing
intesity $I_{\text{out}}$ we get a geometric series

\begin{align*}
  I_{\text{out}} & = I_{\text{in}} T_1 T_2 T_g \sum_{n=0}^\infty R_1^n R_2^n T_g^{2n}\\
  & = I_{\text{in}} T_1 T_2 T_g \cdot \frac{1}{1 - R_1R_2T_g^2},
\end{align*}

where for the last equation we need $R_1R_2T_g^2 < 1$ to hold, which
is clearly true since all entering variables lay between 0 and 1.

With this formula we are able to compute the sample air intensity $I$
as well as the zero air (i.\,e.\ the trace gas free) intensity $I_0$
depending on the different reflectivities and transmittances. To do
this we will use a few approximations, which are listed in the
following

\begin{enumerate}
\item $I_{\text{in}}$ is the same for both $I$ and $I_0$. This means
  we are neglecting any fluctuations in the light source intensity
  coming from temperature instabilities or other optomechanic effects.
\item We assume $R_1 \approx R_2 \eqqcolon R \approx 1$, meaning $(R -
  1) \ll 1$. Since we use hihgly reflective mirrors this assumption
  seems reasonable.
\item We assume $\epsilon \cdot d \ll 1$ for both sample and zero
  air. Furthermore we assume the transmittances for both zero air and
  sample air to be equal to first order. We write $T_g \approx
  T_{g,0}$. This seems also reasonable as the absorption in air is
  comparativly weak and our geometrical pathlength is in the order of
  magnitude of \SI{1}{\meter}.
\item We assume  $(R - 1) + \epsilon d \ll 1$, too. This is only a
  lightly stronger condition then the above mentioned two, but
  necessary. 
\item We assume to be allowed to neglect higher order monomials of the
  form $(R-1)^i(\epsilon d)^j$  with $i+j \geq 2$, $i,j \in \N_0$.
\end{enumerate}

In the following we will always refer to the number of the above
mentioned assumptions, when used.

The information of our trace gas concentration should in this setup
still be obained by comparing $I$ and $I_0$. We still expect an
exponential factor between the two so we introduce the \emph{Cavity
  Enhanced Optical Density} $D_{\text{CE}}$ by

\begin{align}
  I(\lambda) = I_0(\lambda) \cdot \exp(- D_{\text{CE}}) = I_0(\lambda)
  \cdot \exp(-\delta(\lambda) \cdot L_{\text{eff}}(\lambda)),
\end{align}

where $\delta \coloneqq \epsilon - \epsilon_0$ is the difference
between the absorption coefficients of sample and zero air, i.\,e.\ the
absorption coefficient of the trace gases and $L_{\text{eff}}$ is the
wavelength dependent \emph{effective pathlength} of the system. Next
we will take a closer look at $D_{\text{CE}}$.

\begin{align}
  D_{\text{CE}}(\lambda) & \coloneqq \ln\left(
                           \frac{I_0(\lambda)}{I(\lambda)}\right)\nonumber\\
                         & = \ln\left ( \frac{I_{\text{in}}T_1T_2T_{g,0}(1 -
                           (RT_{g,0})^2)^{-1}}{I_{\text{in}}T_1T_2T_g(1 -
                           (RT_g)^2)^{-1}}\right)\nonumber\\
                         & \stackrel{3.}{\approx} \ln\left( \frac{1 -
                           (RT_g)^2}{1 - (RT_{g,0})^2}\right)\label{eq:d_ce},
\end{align}

where the wavelenght dependencies were dropped after the first
equation to preserve clarity. 

To evaluate this expression further we
first have a look at the separate expression $(RT)^2$:

\begin{align}
  [RT]^2 & = [R \exp(-\epsilon d)]^2 \nonumber\\
         & \stackrel{3.}{\approx} [R \cdot(1 - \epsilon d)]^2 \nonumber\\
         & = [(1 + (R - 1))\cdot (1 - \epsilon d)]^2 \nonumber\\
         & \stackrel{5.}{\approx} [1 - (1 - R + \epsilon d)]^2 \nonumber\\
         & \stackrel{4.}{\approx} 1 - 2 \cdot (1 - R + \epsilon d)\label{eq:rt}.
\end{align}

Inserting Equation~\eqref{eq:rt} into Equation~\eqref{eq:d_ce} we
yield

\begin{align}
  D_{\text{CE}} & \approx \ln \left ( \frac{1 - (1 - 2\cdot ( 1- R +
  \epsilon d))}{1 - (1 - 2 \cdot (1 - R + \epsilon_0 d))}\right)\\
  & = \ln \left ( \frac{1 - R + \epsilon d}{1 - R + \epsilon_0
    d}\right) \\
  & = \ln \left ( 1 + \frac{ \delta d}{1 - R + \epsilon_0 d}\right) \quad
    \text{with } \delta \coloneqq \epsilon - \epsilon_0.
\end{align}

This last equation can be reformulated to

\begin{align}
  \exp(D_{\text{CE}}(\lambda)) - 1 = \frac{I_0(\lambda)}{I(\lambda)} -
  1 = \frac{d}{1 - R(\lambda) + \epsilon_0(\lambda) d} \cdot
  \delta(\lambda)\label{eq:i-1}, 
\end{align}

where all the trace gas information is bundled in $\delta(\lambda)$
and the other right hand side parts are trace gas independent.

In a next step we want anlogously $D_{\text{CE}}$ to be given by a
pathlength multiplied by $\delta$. Thus we define

\begin{align}
  L_{\text{eff}}(\lambda) \coloneqq \frac{D_{\text{CE}}(\lambda)}{\delta(\lambda)}.
\end{align}

In this equation we can enter Equation~\ref{eq:i-1} solved for
$\delta$ and get

\begin{align}
  L_{\text{eff}} = \frac{D_{\text{CE}}}{\exp(D_{\text{CE}}) - 1} \cdot
  \underbrace{\frac{d}{1 - R + d\epsilon_0}}_{\eqqcolon L_0}.
\end{align}

We see that with this definition of $L_0$ it is completely independent
of any trace gas influence and hence all the trace gas dependence is
restricted to the first term. Furthermore we see that $L_0$ directly
depends on the mirror reflectivity. All in all $L_0$ depends only on
the geometry of our setup (if we assume $\epsilon_0$ to be fixed) and
we have reached the desired separation of $L_{\text{eff}}$.

Looking only at the definition of $L_0$ and comparing it to
Equation~\eqref{eq:i-1}, we yield

\begin{align}
  \delta \cdot L_0 = \frac{I_0}{I} - 1 \eqqcolon D_{\text{eff}}, \label{eq:ce-central}
\end{align}

where $D_{\text{eff}}$ is calle \emph{Effective Optical Density}.This
is the central equation for the CE-DOAS evaluation. $I$ and $I_0$ are
measured $L_0$ only depends on the geometry, so the only place, where
the trace gas concentrations enter are through $\delta$, which makes
it very easy for us to fit the concentrations.

In addition this equation also allows us to determine the pathlength
$L_0$. Using Helium as sample air we have a rather large difference
and well known $\delta$, which we can then use to fit
$L_0$.\todo{Platt 2009 und Fiedler 2003 aufnehmen}

In the actual evaluation we then take, as in the DOAS evaluation, the fit function

\begin{align*}
  f(\lambda) \coloneqq L_0(\lambda)\cdot\sum_{j=1}^n \sigma_j(\lambda)
  \cdot c_j + \sum_i a_i \lambda^i,
\end{align*}

where the parameters $c_j$ and $a_i$ are fitted to
$D_{\text{eff}}$. This is done using the leas square methods. The
polynomial is again added to compensate for broadband structure.

%%% Local Variables: 
%%% mode: latex
%%% TeX-master: "../Bachelor"
%%% End: 
